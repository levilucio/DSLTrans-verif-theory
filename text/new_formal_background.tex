\section{Formal Background}
\label{sec:formal_background}

In this section we will introduce the formal concepts that will be used throughout all this paper. We start in \cref{sec:typed_graphs} by a few (typed) graph concepts that will be used as mathematical building blocks throughout this paper. In particular we will introduce the notion of typed graph, typed graph union and subset, and useful relations between typed graphs based on homomorphisms. Notice that these concepts are well known from graph theory and are only slightly customized for our purposes.

\reviewer{Your formalisation is made more complex than necessary because you do not put
the power of typed graphs to work. If your source and target metamodels have
disjoint types (which you can assume without loss of generality) you can take
their union, augmented with trace edge types from all target node types to all
source node types, and take that to be the meta-model for an input-output model,
and also for a rule. Suddenly you don’t have to carry arround the distinction be-
tween Match and Apply (for rules) or input and output (for input-output models)
around any more: they are just the projections onto the types of the source, resp.
target metamodel.}

\reviewer{Section 4.2.2 on “backward links” is rather confusing, as it speaks about trace-
ability links being added to rules. You never add traceability links to rules any-
where in the paper; in fact, your formal notion of a rule has no room for them.
Instead, rules only have backward links. (The clarity of the the situation is not
improved by the fact that those are labelled trace.)}

\reviewer{I think the situation would be much more clear if you were to use the common
notion of left hand side and right hand side of a rule. The left hand side of a
rule rl is precisely your ||rl||, the right hand side is the entire rl augmented by
your traceability links, as in 4.2.2. Suddenly we are back in a well-known graph
transformation formalism, and your rules are in fact so simple (no deletion, no
negative application conditions, injective matching) that I believe it makes no
difference whether you use algebraic graph transformation or some more con-
structively defined variant. The only standard notion is that of transitive closure
in the left hand side; but there are plenty of GT tools that offer this extension.
In fact I see no reason why your rule application would then not precisely coin-
cide with the the same notion in, say, SPO graph rewriting; and I hope you agree
that this would help you no end in explaining what you are doing. (If, on the
other hand, there is after all some difference then this, too, would be interesting
to know, as that difference is not at all apparent right now.)}

\reviewer{Viewed like this, moreover, I think your notions of input-output models and rules
are very close to triple graph transformation as used by Schrr at all. I think the
main (if not only) difference lies in the fact that your "glue graph" is actually not
a graph; instead, you use traceability edges directly from the target model to the
source model. If you were to turn those edges into nodes with source and target
edges, even that difference would disappear.}


Armed with the fundamental notion of typed graph, we can then introduce other formal concepts in Sections~\ref{sec:DSLTrans_formal}, \ref{sec:DSLTrans_syntax} and \ref{sec:DSLTrans_semantics} which describe the artifacts from the modeling and transformation world that we require for our verification technique. Naturally, we start by introducing the central notion of \emph{metamodel}, allowing the description of the inputs and outputs of a model transformation. Other fundamental notions we will define in this section are \emph{model}, \emph{transformation rule}, \emph{transformation} and the semantic concept of \emph{model transformation execution}. Several auxiliary and intermediate notions for defining the syntax and semantics of our techniques will also be introduced here. 

Note that this section presents a collection of formal tools that are used in the subsequent sections of this paper where the contributions of this paper are presented. It is meant as a formal reference for the upcoming formal development. This section can be safely skipped or skimmed by the reader, who can return to these definitions punctually to understand the detailed formal underpinning of our approach. 

\reviewer{There is a lot of redundancy in the paper, on several levels:
  - input-output-models, path conditions and properties are
    all defined very similarly. Couldn't you subsume them in a single
    definition and spell out the differences?}


\newcommand{\defineggprime}{Let $\langle V,E,(s,t),\tau,VT,ET\rangle = g$, and\\ $\langle V',E',(s',t'),\tau',VT',ET'\rangle = g'$, where $g, g' \in \textsc{Tg}$.}

\newcommand{\ET}{\mathit{ET}}
\newcommand{\VT}{\mathit{VT}}


\bentley{Make sure that terminology is consistent between elements and vertices}

\subsection{Graph Structures and Functions}
\label{sec:typed_graphs}


\subsubsection*{Typed Graph}
We will start by introducing the notion of typed graph. A typed graph is the essential object we will use throughout our mathematical development. Typed graphs will be used to formalise all the important graph-like structures we will present in this paper. A typed graph is a directed multigraph (a graph allowing multiple edges between two vertices) where vertices and edges are typed.


\begin{definition}{Typed Graph\\}
\label{def:typed_graph}
A typed graph is a 6-tuple $\langle V,E,(s,t),\tau, \VT, \ET\rangle$ where:
\begin{itemize}
\item $V$ is a finite set of vertices
\item $E$ is a finite set of directed edges connecting the vertices $V$
\item $(s,t)$ is a pair of functions $s: E\rightarrow V$ and $t: E\rightarrow V$ that respectively provide the source and target vertices for each edge in the graph
\item Function $\tau:V\cup E\rightarrow \VT \cup \ET$ is a typing function for the elements of $V$ and $E$, where $\VT$ and $\ET$ are disjoint finite sets of vertex and edge type identifiers and $\tau(v)\in \VT$ if $v\in V$ and $\tau(e)\in \ET$ if $e\in E$
\item Edges $e\in E$ are noted $v\xrightarrow{e} v'$ if $s(e)=v$ and $t(e)=v'$, or simply e if the context is unambiguous
\item The set of all typed graphs is called $\textsc{Tg}$. 
\end{itemize}
\end{definition}


\subsubsection{Vertex and Edge Types}

Note that the our verification technique is geared toward model-driven engineering. Therefore, the types of vertices and edges in our typed graphs will be drawn from a particular \textit{metamodel}.

Sample metamodels for our running example can be seen in figures above \bentley{Fix.}. The particular metamodel for a typed graph will be indicated in the formalization when required, as (potentially) different metamodels will be present for the input and output graphs in DSLTrans rules.

We assume the existence of a number of functions in our formalization to aid the treatment of typing:

\begin{itemize}
\item $\mathit{isAbstract}: \mathit{VT} \rightarrow \{\mathit{true}, \mathit{false}\}$
\begin{itemize}
\item Where $\mathit{isAbstract}(\VT)$ returns $\mathit{true}$ iff $\VT$ is denoted as abstract (not able to be instantiated) by the metamodel, else $\mathit{false}$
\end{itemize}

\item $\mathit{isIndirect}: \mathit{ET} \rightarrow \{\mathit{true}, \mathit{false}\}$
\begin{itemize}
\item Where $\mathit{isIndirect}(\ET)$ returns $\mathit{true}$ iff $\ET$ is denoted as an \textit{indirect edge}, else $\mathit{false}$. The \textit{indirect} classification allows matching over indirect paths between vertices. This will be further clarified when required for our constructs.
\end{itemize}

\item $\mathit{matchesOver}: \{\VT \cup \ET\} \times \{\VT \cup \ET\} \rightarrow \{\mathit{true}, \mathit{false}\}$

\begin{itemize}
\item The purpose of this function is to assist in handling polymorphism in the metamodel
\item $\mathit{matchesOver}(T, T) \rightarrow \mathit{true}$
\item $\mathit{matchesOver}(T, T') \rightarrow \mathit{true}$ iff T' is a subclass of T in some defined partial ordering $\leq$ given by the metamodel
\item Otherwise, $\mathit{false}$
\end{itemize}
\end{itemize}

For simplification purposes, we will not model the edge cardinalities given by a metamodel in our notion of typed graph. Note that in fact require these cardinalities to be relaxed to perform our graph rewriting.

\subsubsection*{Typed Subgraph}
We now define the useful notion of typed subgraph. As expected, a typed subgraph is simply a restriction of a typed graph to some of its vertices and edges. 

\begin{definition}{Typed Subgraph\\}
\label{def:typedsubgraph}
\defineggprime

$g'$ is a typed subgraph of $g$, written $g'\sqsubseteq g$, iff $V'\subseteq V$, $E'\subseteq E$ and $\tau'=\tau |_{V'\cup E'}$.

\end{definition}

Given a typed graph $g\in \textsc{Tg}$, will use the notation $Components(g)$ to describe the set of strongly connected typed graphs in $g$. Also, we will use the notation $g|_{t}$ to refer to the restriction of graph $g$ to its subgraph containing only edges of type $t$.


\subsubsection*{Typed Graph Union}
We now define how two typed graphs are united. A union of two typed graphs is trivially the set union of all the components of those two typed graphs. Note that we do not require the components of the two graphs to be disjoint, as in the following joint unions will be used to merge typed graphs.

\begin{definition}{Typed Graph Union\\}
\label{def:typed_graph_union}

\defineggprime

%where $VT$ and $ET'$ are disjoint sets, as well as $VT'$ and $ET$.
The typed graph union is the function $\sqcup :\textsc{Tg}\times \textsc{Tg}\rightarrow \textsc{Tg}$ defined as:
\begin{multline*}
\big\langle V,E,(s,t),\tau,\VT,\ET\big\rangle\;\sqcup\;\big\langle V',E',(s',t'),\tau',\VT',\ET'\big\rangle=\\\big\langle V\cup V', E\cup
E',(s\cup s', t\cup t'), \tau\cup \tau', \VT\cup \VT', \ET\cup ET'\big\rangle
\end{multline*}
\end{definition}

Note: as a reviewer helpfully pointed out, we require that $s \cup s'$, $t \cup t'$, and $\tau \cup \tau'$ coincide on common elements. However, this can be assumed w. l. o. g..

\subsubsection*{Typed Graph Homomorphism}
For the formal development of our technique, we are interested in relations between typed graphs that are structure-preserving, i.e. homomorphisms. Homomorphisms between typed graphs preserve not only structure, but also the types of vertices and edges that are mapped. Note that, trivially, a typed graph homomorphism is a graph homomorphism.

\begin{definition}{Typed Graph Homomorphism\\}
\label{def:typed_graph_homomorphism}
\defineggprime

A typed graph homomorphism between $g$ and $g'$ is a function $f: f_v \cup f_e$ such that:
\begin{itemize}
\item $f_v: V\rightarrow V'$
\item $f_e: E\rightarrow E'$
\item $\forall v_1 \xrightarrow{e} v_2\in E$:
\begin{itemize}
\item Let $ v_1' = f_v(v_1), v_2' = f_v(v_2), e' = f_e(e)$
\item $v_1', v_2' \in V', e' \in E'$
\item $s'(e') = v_1'$, $t'(e') = v_2'$
\item $\mathit{matchesOver}(\tau(v_1), \tau(v_1') \land \\\mathit{matchesOver}(\tau(v_2), \tau(v_2') \land\\ \mathit{matchesOver}(\tau(e), \tau(e')$
\end{itemize}

%, where $\tau(v_1)=\tau'(f(v_1))$, $\tau(v_2)=\tau'(f(v_2))$ and also $\tau(e)=\tau(e')$.
\item The domain of $f$ is noted $Dom(f)$ and the co-domain of $f$ is noted $CoDom(f)$.
\end{itemize}  
\end{definition}

\bentley{Maybe put `inj' and `sub' in the homomorphism to reduce number of symbols}
 When an \emph{injective} typed graph homomorphism $f$ exists between $g$ and $g'$ we write $g \stackrel{f}{\vartriangleleft} g'$, or simply $g \vartriangleleft g'$ when the context is unambiguous. When a \emph{surjective} typed graph homomorphism $f$ exists between typed graphs $g$ and $g'$ we write $g \stackrel{f}{\blacktriangleleft} g'$, or also simply $g \blacktriangleleft g'$ in an unambiguous context. 

\subsubsection*{Typed Graph Edge Homomorphism}

Our technique also requires a form of graph homomorphism that primarily focuses on matching the edges of the graphs. This homomorphism will be an injective match regarding the edges, but note that a particular vertex in the source graph may match onto multiple vertices in the target graph. \bentley{What's the term for this?}

\begin{definition}{Typed Graph Edge Homomorphism\\}
\label{def:typed_graph_edge_homomorphism}
\defineggprime

A typed graph edge homomorphism between $g$ and $g'$ is a function $h: h_v \cup h_e$ such that:
\begin{itemize}
\item $h_v: V'\rightarrow V$
\begin{itemize}
\item Note that $h_v$ is an function from a vertex in the target graph onto a vertex in the source graph
\end{itemize}
\item $h_e: E\rightarrow E'$
\item $\forall v_1 \xrightarrow{e} v_2\in E$:
\begin{itemize}
\item Let $e' = h_e(e), v_1' = s'(e'), v_2' = t'(e')$
\item $v_1', v_2' \in V', e' \in E'$
\item $v_1 = h_v(v_1'), v_2 = h_v(v_2')$
\item $\mathit{matchesOver}(\tau(v_1), \tau(v_1') \land \\\mathit{matchesOver}(\tau(v_2), \tau(v_2') \land\\ \mathit{matchesOver}(\tau(e), \tau(e')$
\end{itemize}

\end{itemize}  
\end{definition}

Again, we wish to highlight the fact that this homomorphism is different than Definition~\ref{def:typed_graph_homomorphism} in the vertex matching function. A vertex in the target graph matches onto a vertex in the source graph, with multiple target vertices matching onto the same source vertex. 
\bentley{Create diagram to illustrate}

This unusual homomorphism allows our technique to `split' our source graph over multiple locations in the target graph. This is critical, as our target graphs are constructed in a way that allows for vertex duplication. That is, two vertices of a particular type in the target graph may only represent one underlying element, as discussed in Section~\ref{sec:abstraction_relation}. Constructing this morphism allows our prover implementation to avoid explicitly creating `disambiguated' graphs where these duplications are resolved.


\subsection*{Typed Graph Isomorphism}
Two typed graphs are said to be isomorphic if they have exactly the same shape and related vertices and edges have the same type.


\begin{definition}{Typed Graph Isomorphism\\}
\label{def:typed_graph_isomorphism}
\defineggprime

We say that $g$ and $g'$ are isomorphic, written $g\cong g'$, iff there exists a bijective typed graph homomorphism $f:V\rightarrow V'$ such that $f^{-1}:V'\rightarrow V$ is a typed graph homomorphism.
\end{definition}




\subsubsection*{Transitive Closure}

DSLTrans matching constructs allow for the matching of indirect links, where as described in Section~\ref{subsec:DSLTrans_constructs}, rules may match over indirect paths between elements. This transitive closure allow for the explicit creation of all edges implied by these indirect links to aid in matching.


\begin{definition}{Transitive Closure\\}
\label{def:instance_closure}

Let $g$ be a graph $\langle V,E,(s, t),\tau\rangle \in \textsc{Tg}$. Then the transitive closure $g^{*} = \langle V,E',(s, t),\tau\rangle \in \textsc{Tg}$, where:

$E' = E \cup $ the transitive closure of $\big\{v\xrightarrow{e}v' | \mathit{isIndirect(e)} \big\}$
\end{definition}


Given a graph, its transitive closure includes, besides the original graph, all the edges belonging to the transitive closure of indirect links in that graph. Note that these transitive edges are given by the function $\mathit{isIndirect}$.

In the definitions that follow we will use the $*$ notation, as in \cref{def:instance_closure}, to denote the transitive closure of our structures.


\subsection{DSLTrans Constructs}
\label{sec:DSLTrans_formal}



This section will detail the abstract syntax of the constructs involved in a DSLTrans transformation.

As discussed in Section~\ref{subsec:DSLTrans_syntax}, DSLTrans transformations are composed of rules arranged in layers.


\subsubsection*{DSLTrans Transformation Rule}

A transformation rule is the elemental block of a DSLTrans transformation. Several transformation rules can be observed in the Police Station transformation in \cref{fig:dsltransformation}.

A transformation rule includes a non-empty match pattern and a non-empty apply pattern. This is also known in the model transformation literature as a rule's \emph{left hand side} and \emph{right hand side}. A match pattern can include indirect links that are used to transitively match containment relations in a model. The apply pattern of a rule always contains at least one apply element that is not connected to a backward link or an edge, meaning in practice that a rule will always produce something and not only match. An apply pattern does not include indirect links as it is used only for the construction of parts of instances of a metamodel.

A transformation rule also includes backward links, as informally introduced in \cref{subsec:DSLTrans_constructs}. As described in that section, backward links define dependencies between rules. \bentley{Expand}


\begin{definition}{DSLTrans Transformation Rule\\}
\label{def:transformation_rule}

A DSLTrans transformation rule is a three-tuple $\big\langle \mathit{Match}$, $\mathit{Apply}$, $\mathit{backward}\big\rangle$, where:

\begin{itemize}
\item $\mathit{Match}, \mathit{Apply} \in \textsc{Tg}$
\item $\mathit{Match}, \mathit{Apply} $ are non-empty and disjoint
\item $\mathit{backward} = \{E_{back}, (s_{back}, t_{back})\}$
\begin{itemize}
\item $E_{back} \cap E_{Match} = \emptyset, E_{back} \cap E_{Apply} = \emptyset$
\end{itemize}
\end{itemize}  

Note that when we require an element of the $\mathit{Match}$ or $\mathit{Apply}$ graphs, such as the vertices, we will index the required element. For example, the vertices for the \textit{Match} graph will be $V_{\textit{Match}}$.

\begin{itemize}
\item $E_{back}$ contains the backward links
\item $(s_{back}, t_{back})$ is a pair of functions $s_{back}: E_{back}\rightarrow V_{\textit{Apply}}$ and $t_{back}: E_{back}\rightarrow V_{\textit{Match}}$ that respectively provide the source and target vertices for each backward link
\begin{itemize}
\item Note the source of backward links is a vertex in $V_{\textit{Match}}$ while the target is a vertex in $V_{\textit{Apply}}$
\end{itemize}
\end{itemize}

\end{definition}

We additionally impose that for the well-formedness of a DSLTrans rule, there always exists an element or edge to be created in the \textit{Apply} graph to be created. That is, either there are edges to be created in the \textit{Apply} graph, or there is a vertex in the \textit{Apply} graph that is not the source of a backward link.

\begin{itemize}
\item $E_{\textit{Apply}} \neq \emptyset \lor$
\item $ \exists v \in V_{\textit{Apply}} | \big\{\forall e \in E_{back}: s_{back}(e) \neq v \big\}$
\end{itemize}


\subsubsection*{Matcher of a Transformation Rule}

To create the actual matcher construct that will be matched during a rule's execution, the rule's \textit{Match} graph is combined with backward links, as well as any \textit{Apply} vertices connected to the backward links.


\begin{definition}{Matcher of a Transformation Rule\\}
\label{def:back_match_transformation_rule}

Let the transformation rule $r = \big\langle \mathit{Match}$, $\mathit{Apply}$, $\mathit{backward}\big\rangle$.

We define $r$'s matcher, noted $\Vert r \Vert$, to be a typed graph six-tuple $\langle V,E,(s,t),\tau, \VT, \ET\rangle$, where:
\begin{itemize}
\item $V = V_{Match} \cup \big\{v \in V_{Apply} | \{\exists e \in E_{back} | s_{back}(e) = v\}\big\}$
\item $E = E_{Match} \cup E_{back}$
\item $s = s_{Match} \cup s_{back}$, $t = t_{Match} \cup t_{back}$
\item $\tau = \tau_{Match} \cup \tau_{Apply}$
\item $\VT = \VT_{Match} \cup \VT_{Apply}$
\item $\ET = \ET_{Match} \cup \ET_{Apply}$
\end{itemize}

\end{definition}

\subsubsection*{DSLTrans Transformation Layer}

\begin{definition}{Layer, Model Transformation\\}
\label{def:layer_transformation}

A layer is a finite set of transformation rules $l\subseteq \textsc{Rule}^{sr}_{tg}$. The set of all layers for a source metamodel $sr$ and a target metamodel $tg$  is called $\textsc{Layer}^{sr}_{tg}$. A model transformation is a finite list of layers denoted $[l_{1}::l_{2}::\ldots::l_{n}]$ where $l_{k}\in \textsc{Layer}^{sr}_{tg}$ and $1\leq k \leq n$, $n\in \mathbb{N}$. We also impose that for any pair of rules $rl_1,rl_2\in \bigcup_{1\leq k\leq n}l_k$, if $\Vert rl_1\Vert \cong rl$ and $rl\sqsubseteq \Vert rl_2\Vert $ then $rl_2$ appears in a layer later than $rl_1$ and the apply parts of $rl_1$ and $rl_2$ are not isomorphic. The set of all transformations for a source metamodel $s$ and a target metamodel $t$ is called $\textsc{Transf}^{sr}_{tr}$. 

% The expansion of a layer $l\subseteq RULE^{sr}_{tg}$, noted $l^{\star}$, is such that that if
% $rl\in l$ then $rl'\in l$ if $rl'\in rl^{\star}$. This means that if a
% transformation rule is present is the layer, its expansion is also part of the
% layer.

\end{definition}

\reviewer{The second half of Def. 16 raises a lot of questions. First of all, neither $\sqsubseteq 	 $ nor
$\cong$ are defined over rules; however, I can imagine what it would look like. Then,
what role does rl play; do you just mean $||rl_1 || \sqsubseteq ||rl_2 ||$ given a suitable isomorphic copy? What if $||rl_1 || \cong ||rl_2 || $(as a special case); is your condition then not
unsatisfiable as $rl_1$ must appear after $rl_2$ as well as vice versa?}


\reviewer{More importantly though, why do you care? Your semantics makes sure that
rules are never considered for application twice in succession, so dependencies
are always resolved unambiguously.}


\cref{def:layer_transformation} formalises the abstract syntax of a model transformation, introduced in \cref{sec:dsltrans}. An example of a model transformation can be observed in \cref{fig:dsltransformation}, the Police Station transformation. As expected, a formal DSLTrans transformation is composed of a sequence of layers where each layer is composed of a set of rules. The last condition of \cref{def:layer_transformation} imposes that, for any two pair of rules in the transformation, the matcher of the second rule never partially or totally subsumes (or contains) the matcher of the first rule, unless the second rule is in a subsequent layer and produces something more than the first rule. This condition avoids situations where the execution of a rule in a DSLTrans model transformation necessarily implies the execution of another rule (except for when rules having backward links necessarily execute because all their dependencies were created during the execution of previous layers). \reviewer{I do not fully understand why this
  could be problematic. Do you have an example?}

\paragraph{\textbf{Notation:}}


We naturally extend to transformation rules (\cref{def:transformation_rule}) and transformation executions (\cref{def:modeltransformation}) the typed graph operations introduced in \cref{sec:formal_background}.  Also, given a structure such as transformation rule $rl = \langle V,E,st,\tau,Match,\\Apply\rangle$, we will refer to the structure's components by using the component's name followed by the variable that holds the structure in between parenthesis. For example, we will write $V(rl)$ to designate the $V$ component of $rl$ or $Apply(rl)$ to designate $rl$'s $Apply$ component.



\begin{definition}{Input-Output Model\\}
\label{def:input_output_model}
An input-output model is a 6-tuple $\big\langle V,E,(s,t),\tau,Input,\\Output\big\rangle$, where:
\begin{itemize}
\item $Input=\langle V',E',st',\tau'\rangle\in \textsc{Instance}^{sr}$ is a model
\item $Output=\langle V'',E'',st'',\tau''\rangle\in \textsc{Instance}^{tg}$ is a metamodel instance
\item $Input$ and $Output$ are disjoint
\end{itemize}

Additionally we have that $V=V'\cup V''$, $E\subseteq E'\cup E''$ and $\tau\subseteq \tau'\cup \tau''$, where the co-domain of $\tau$ is the union of the co-domains of $\tau'$ and $\tau''$ and the set $\{trace\}$.

An edge $e \in E\setminus E'\cup E''$ is called a \emph{traceability link} and is such that $s(e)\in V''$, $t(e)\in V'$ and $\tau(e)=trace$. 

 set of all match-apply patterns for a source metamodel $sr$ and a target metamodel $tg$ is called $\textsc{Iom}^{sr}_{tg}$.
\end{definition}

\reviewer{Why is Input a model and Output a metamodel instance? Note that
$\backslash$ binds stronger than $\cup$ in $E \backslash E' \cup E$ ′′ . “match-apply patterns” → “input-output
patterns”}
%Change if needed
% \levi{A match-apply model is the union of an input and output models, plus potential traceability links between those models. Traceability links are now typed using the typing function $\tau$. A traceability link it typed with type ``trace''. This makes the definitions shorter and more compact, which was mentioned by one of the reviewers of the paper. However, the fact that a match-apply model is now made up of models makes it that the input and the output of the transformation are always well-formed, meaning that the containment relation is always respected. It may be that a transformation does not always produce a model respecting the containment relation, which is a problem. This can be solved however if we only consider that well-formed transformations actually have semantics.}

An input-output model is an object we will use when defining the semantics of a DSLTrans model transformation in \cref{sec:DSLTrans_formal}. It is composed of two metamodel instances, one called the \emph{input} and the other one the \emph{output}. An input-output model allows the representation of intermediate operational states during the execution of a model transformation. It may include a particular type of edges called \emph{traceability links}, for keeping a history of which elements in the output model originated from which elements in the input model.


\subsection{Semantic Transformation Constructs}
\label{sec:DSLTrans_semantics}

In the definition that follows we introduce the notion of execution of a DSLTrans model transformation. For our purposes it is sufficient to introduce it as an input-output model (see \cref{def:input_output_model}), containing the input model for the transformation, the produced output, and the traceability links built during execution. Due to space limitations, we cannot introduce the semantics of DSLTrans in the main text of this paper. We thus refer the reader to \cref{sec:DSLTrans_formal} for a formal description on how DSLTrans transformation executions are built. 

\reviewer{Definition 17: There is a copy/paste error w.r.t. Def. B.16 (which only serves to
strengthen my point, made above, about avoiding this kind of duplication). In
this case, B.16 is the correct version.}

\begin{definition} {Model Transformation Execution\\}
\label{def:modeltransformation} 

Let  $tr\in \textsc{Transf}^{sr}_{tg}$ be a transformation and $input \in \textsc{Model}^{sr}$ be a model. Assume we also have that: $$\langle V,E,st,\tau,input,\epsilon\rangle,tr \xrightarrow{trstep} \langle V',E',st',\tau',input,output\rangle$$
A model transformation execution is the input-output model $\langle V',E',st',\tau',input,output\rangle\in \textsc{Iom}^{sr}_{tg}$, where $output\in \textsc{Iom}^{sr}_{tg}$ is an input-output model. The set of all model transformation executions for transformation $tr$ is written $\textsc{Exec}(tr)$. A model transformation with an empty input model is noted $\epsilon_{ex}$. Note that relation $trstep$ is formally defined in \cref{sec:DSLTrans_formal_appendix}.
\end{definition}


Finally, as stated in \cref{def:modeltransformation}, we consider a model transformation execution to be the input-output model (IOM) resulting from executing a set of rules on a starting IOM. This starting IOM includes the transformation's input in its input part and has an empty output part. The starting IOM represents the first step of the transformation when no rule has been executed yet. A transformation execution results from executing all the rules in a DSLTrans model transformation.

\subsubsection{Transformation Language Semantics}

We will now address the semantics of the DSLTrans language. We will start by defining a match function that, given an input-output model and a transformation rule, returns all subgraphs of that input-output model where the rule's match pattern (including backward links) is found.  

% \levi{Need to be careful throughout all this section to understand whether we model the match and apply functions as functions or relations. Because of the non-deterministic generation of new vertices in the definition of apply function~\ref{def:apply_function}, relations are used in the definitions that follow to evaluate a transformation. This is an issue of the formalism used to model, should we make it explicit?}

\begin{definition}{Match Function}
\label{def:match_function}

Let $m_{in}\in \textsc{Iom}^{sr}_{tr}$ be a input-output model and $rl\in \textsc{Rule}^{sr}_{tg}$ be a transformation
rule. The $match : \textsc{Iom}^{sr}_{tg}\times \textsc{Rule}^{sr}_{tg}\rightarrow
\mathcal{P}(\textsc{Iom}^{sr}_{tg})$ function is defined as follows: $$match_{rl}(m_{in})= \big\{ glue_{noind}\;|\;
glue\sqsubseteq m_{in}^{*} \land glue \cong \Vert rl\Vert\big\}$$

where $glue\in \textsc{Iom}^{sr}_{tr}$ is an input-output model and $glue_{noind}$ is a version of $glue$ where the indirect links have been removed.
% \levi{all backward links are replaced by trace links such that the matching using subgraph isomorphim can work.}
% 
% \levi{Where $rl^{-}$ is rule $rl$ without nodes in the apply part that are not connected to \emph{trace links}, as well as links adjacent to those nodes. $mg^{-ind}$ is the match graph without indirect links that come from the rule, given those cannot be part of the intermediate $MAM$ match result since that result is going to be united with the original $MAM$ (this was incorrect in the original SLE paper).}
% \levi{The matcher function returns all graphs of the input MAM that are matched by a rule, including its backward links. Note that to find a subgraph ($\sqsubseteq$) of the input MAM that is isomorphic ($\cong$) to the rule being matched because the graph rewriting is achieved by joint union of the match graph extended by the new rewrite node, and the original MAM.}
% \levi{TODO: review the $mg^{-ind}$ and the $rl^{-}$ notations.}

% Due to the fact that the $\cong$ relation is based on the notion of graph
% isomorphism, permutations of the same match result may exist in the
% $\big\{g\;|\; g\lhd m \land g \cong strip(tr)\big\}$ set. The --- undefined ---
% $remove:\mathcal{P}(TR^{s}_{t})\rightarrow \mathcal{P}(TR^{s}_{t})$ function is such that
% it removes such undesired permutations.

%The $strip:TR^{s}_{t}\rightarrow TR^{s}_{t}$ function is such that
%$$strip(\big\langle V,E,T,\langle V_m,E_m,T_m,s\rangle,\langle V_a,E_a,T_a,t\rangle,Bl,Il\big\rangle) = \big\langle V',E',T,\langle V_m,E_m,T_m\rangle,\langle V'_a,E'_a,T_a\rangle,Bl,Il\big\rangle$$
%\begin{center}
%where $(v'\in V'_a\Rightarrow v\rightarrow v'\in Bl)\;\land\;(v\rightarrow v'\in E'_a\Rightarrow (v\rightarrow v'\in E_m \land \{v,v'\}\subseteq V'_a))$
%\end{center}
\end{definition}

\reviewer{Def. B.12. First of all, I wonder why you define match as a function rather than
just defining what a match is; namely, an element of the set you are constructing
here. Secondly (and more importantly), why do you not use the notion of homo-
morphism instead of explicitly talking about ismorphic subgraphs? Your way,
you lose the connection between glue and $||rl||$: you know they are isomorphic,
but what if $||rl||$ has symmetries - then there is more than one match, which in
your definition cannot be distinguished.}

\reviewer{You pay the price for this loss of information in Def. B.13, which now is broken: $a\Delta$ depends on $m_{glue} $(which was called glue in Def. 12, why the change in
notation?) in a way depending on the mapping from $m_{glue}$ to $rl$ which you have
discarded.}

\reviewer{Def. B.13 is broken in another way as well: it does not properly define E' but
merely imposes some requirements that are also fulfilled by choosing E' = E.
Instead I know from the discussion about traceability links in Section 4 (which
logically comes after this appendix!) that you really want to add such links for
all newly created nodes of the output graph to all pre-existing nodes of the input
graph; but that is not made clear anywhere at this point.}


The match function in \cref{def:match_function} looks for subgraphs ($glue\sqsubseteq m_{in}^{*}$) of an input-output model that are isomorphic to the backward matcher of the given transformation rule ($glue \cong \Vert rl\Vert$). Note that the containment transitive closure of the input-output model ($m_{in}^{*}$) is considered such that indirect links in the rule can looked for in the input model. Additionally, indirect links need to be removed from the input-output models resulting from the match function ($glue_{noind}$). This is so because indirect links are not part of the original input model, but rather an auxiliary structure.

Let us now turn our attention to the apply function in \cref{def:apply_function}. Its role is to extend all model fragments found by matching a rule on a given input-output model, such that each of those fragments becomes isomorphic to the complete rule (minus its backward links). This process effectively creates the new objects and relations specified in the apply part of the rule, for each of the fragments found when matching the rule.

\begin{definition}{Apply Function}
\label{def:apply_function}

\reviewer{In your definitions you rely on the $\sqcup$ operator (sqare union),
  the definition of which is not entirely clear.  For instance, I do not know where the operator used in Def. B13
  is defined.}
  
Let $m_{glue}\in \textsc{Iom}^{sr}_{tg}$ be a input-output model and $rl\in \textsc{Rule}^{sr}_{tg}$ a
transformation rule. The $apply : \textsc{Iom}^{sr}_{tg}\times \textsc{Rule}^{sr}_{tg}\rightarrow \textsc{Iom}^{sr}_{tg}$ function
is defined as follows: 

$$apply_{rl}(m_{in})=\bigsqcup_{m_{glue}\in match_{rl}(m_{in})}trace_{a_{\Delta}}(m_{glue}\sqcup a_{\Delta})$$

\begin{center}
where $a_{\Delta}\in \textsc{Iom}^{sr}_{tg}$ is such that $m_{glue} \sqcup a_{\Delta}\cong rl_{noind}$.\\
\end{center}

We impose that any instance of $a_\Delta$ is always disjoint from the $m_{in}$ input-output model and also that any two instances of $a_\Delta$ used in the large union are always disjoint.

Partial function $trace:\textsc{Iom}\times \textsc{Iom} \nrightarrow \textsc{Iom}$ is such that $trace_{\langle V_{\Delta},E_{\Delta},st_{\Delta},\tau_{\Delta}\rangle}(\langle V,E,st,\tau\rangle) = \langle V,E',st',\tau'\rangle$ where we have that $E\subseteq E'$, $st\subseteq st'$ (using a light notational abuse for the $s\subseteq s'$ and $t\subseteq t'$), $\tau \subseteq \tau'$ and if $v_1\xrightarrow{e} v_2\in E'\setminus E$ then $v_1\in Output(V_{\Delta})$, $v_2\in Input(V)$ and $\tau'(e)=trace$. Finally, $rl_{noind}$ is a version of $rl$ where indirect links have been removed.
% More concretely, if we have $mam = $\big\langle V,E,st,\tau, Input,Output\big\rangle\in MAM^{sr}_{tg}$ where $Input=\langle V',E',st',\tau'\rangle\in MODEL^{sr}$ and $Output=\langle
% V'',E'',st'',\tau''\rangle\in MODEL^{tg}$, then $trace(mam) = \big\langle V,E,st,\tau''',Input,Output\big\rangle$ where $E'\subseteq E$ and $v\rightarrow E'\setminus E$ 
\end{definition}
% 
% it builds traceability edges between vertices of the output model that were newly built by the $apply$ function (the vertices in ${mg}_{\Delta}$) and all the vertices from the input part of the input-output model. It is necessary to remove the indirect links from the rule because otherwise they would be created during the rewriting by the apply function. 

In \cref{def:apply_function} $a_\Delta$ is an input-output model that contains an instance of the target metamodel. These instances are created by rule $rl$ and are used to extend the sub-models found by the match function. The $trace$ function builds traceability edges between newly created vertices of the output model in $a_{\Delta}$ and all the vertices from the input part of a model fragment found by the match function.

Note that, because we do not pose any constraints on $a_\Delta$ other than the fact that its union with the sub-model $m$ is isomorphic to $rl_{noind}$, the $a_\Delta$ variable can always be satisfied by an unlimited amount of input-output models. In order to avoid an infinite amounts of results when a transformation rule is executed, in what follows we will consider transformation results differ only up to typed graph isomorphism.   

%  Note that because the vertices are not disjoint, when the graph produced by the apply function is united with the original match apply model the new vertices are created. It was necessary to remove the indirect links from the rule during rewriting, because otherwise those indirect links would also be rewritten.

Definitions~\ref{def:match_function} and~\ref{def:apply_function} are complementary: the former gathers all the fragments of an input-output model that are matched by a transformation rule; the latter glues on the output part of each of those fragments new objects and relations created by a transformation rule. 

%  The $strip$ function is used to
%  enable matching over backward links but not elements to be created by the
%  transformation rule. The $back$ function connects all newly created vertices
%  to the elements of the source model that originated them.

%\begin{definition}{Transformation}

%Let $g\in TG$ be a typed graph. The graph transform function $transform : TR\times TG\rightarrow TG$ is recursively defined as:
%\begin{gather*}
%  transform_{(m,a)}(g) =
%  \begin{cases}
%    g  & \text{if } match_{m}(g)=\emptyset \\
%    instance(a)\cup transform_{(m,a)}(g') &\text{if } match_{m}(g)\neq \emptyset
%  \end{cases}
%\end{gather*}
%$$\text{where } g'=mark_{(m)}(g)$$
%\end{definition}

%\subsection{Transformation Semantics}
%\begin{definition} {Layer Semantics}

%Let $l\in Layer$ be a layer and $\{m,m'\} \in MAM^{s}_{t}$ be a match
%$$\frac{}
%{m,\emptyset \stackrel{layerstep}{\rightarrow}m}$$
%$$\frac{tr\in layer,\; transform_{tr}(m) = m''\; m'',layer\backslash\left\{tr\right\} \xrightarrow{layerstep}	 m'}{m,layer \xrightarrow{layerstep} m'}$$
%\end{definition}

\begin{definition} {Layer Step Semantics}
\label{def:layer_step_semantics}

Let $l\in \textsc{Layer}^{sr}_{tg}$ be a Layer. The \emph{layer step relation}
$\stackrel{layerstep}{\rightarrow}\subseteq \textsc{Iom}^{sr}_{tg}\times \textsc{Iom}^{sr}_{tg} \times \textsc{Layer}^{sr}_{tg}\times
\textsc{Iom}^{sr}_{tg}$ is defined as follows:

$$\frac{}
{\langle m_{in},m_{glue},\emptyset\rangle \xrightarrow{layerstep}m_{in} \sqcup m_{glue}}$$

\begin{multline*}
\frac{\begin{array}{ll} rl\in l,\; apply_{rl}&(m_{in}) = m_{rout},\\
&\langle m_{in},m_{glue}\sqcup m_{rout},l\backslash \{rl\}\rangle \xrightarrow{layerstep} m_{out}
\end{array}}
{\langle m_{in},m_{glue},l\rangle \xrightarrow{layerstep} m_{out}}$$
\end{multline*}

\begin{center}
where $m_{rout} \in\textsc{Iom}^{sr}_{tg}$ and $rl\in \textsc{Rule}^{sr}_{tg}$.
\end{center}
 
We impose that all input-output models that are part of $rout$ and have been generated by rule $rl$ are disjoint from input-output models accumulated in $m_{glue}$ that have been generated by other rules.

\end{definition}

\reviewer{Def. B.14: It seems to me rather roundabout to sequentially apply all rules, ac-
cumulate the results and then take their union. Why do you not take the union
simultaneously with the parallel application of all rules (of a given layer), as
you are essentially doing for all applications of a single rule in Def. B.13? This
would get rid of the artificial sequential ordering, and the whole issue of con-
fluence would not even have to be raised as there is no non-determinism any
more.}

\reviewer{In the constructions defined in Appendix B, you ignore the required absence of
containment cycles. Might such cycles not appear in the union of the graphs,
even if the individual graphs are cycle-free?}

In \cref{def:layer_step_semantics} we build the result of executing a layer of a DSLTrans transformation. The operational semantics-like rules in the definition execute each rule $rl$ in layer $l$, in a non-deterministic order, by using the $apply$ function. The result of executing each rule is accumulated in the temporary $m_{glue}$ input-output model. Finally, when the set of transformation rules in the layer has been exhausted, the result of executing all the rules in the layer (now contained in $m_{glue}$) is united with the input-output model $m_{in}$, the input to layer $l$. Note that this final union produces the result we expect because of the fact that the $m_{glue}$ input-output model is not disjoint from $m_{in}$. The common ``glue'' parts of $m_{glue}$ that have been built by the match function and extended by the apply function are now used to built the result of executing layer $l$.

\cref{def:layer_step_semantics} is the core of DSLTrans' semantics. Many model transformation languages are based on graph rewriting, where the result of each rule rewrite is immediately usable by all other rules. In DSLTrans the result of executing one layer in DSLTrans is totally produced before the input to the layer is changed. This is enforced in \cref{def:layer_step_semantics} by the fact that the apply function always executes over the same $m_{in}$ input-output model and all the results of rule execution in the same layer are added to the $m_{glue}$ structure that is write-only. Rules belonging to the same layer are thus forced to execute independently.


\begin{definition} {Transformation Step Semantics}
\label{def:transformation_step_semantics}

Let $[l::tr]\in \textsc{Transf}^{sr}_{tg}$ be a transformation, where $l\in
\textsc{Layer}^{sr}_{tg}$ is a Layer and $tr$ also a transformation. The \emph{transformation step relation}
$\stackrel{trstep}{\rightarrow}\subseteq \textsc{Iom}^{sr}_{tg}\times \textsc{Transf}^{sr}_{tg}\times
\textsc{Iom}^{sr}_{tg}$ is defined as follows: 
$$\frac{}{\langle m,[]\rangle \xrightarrow{trstep} m}$$

$$\frac{\big\langle m_{in},\epsilon,l^{\star}\big\rangle \xrightarrow{layerstep} m_{inter},\; \langle m_{inter},R\rangle \xrightarrow{trstep} m_{out}}{\langle m_{in},[l::T]\rangle \xrightarrow{trstep} m_{out}} \hspace{.3cm}$$

$$\text{where } l^{\star}=\bigcup_{rl\in l}rl^{\star}$$
\end{definition}

\reviewer{I believe tr, T and R in this definition stand for the same thing.}
While the execution of the rules belonging to a layer happens in parallel, the execution of the layers of a transformation happens sequentially. As per \cref{def:transformation_step_semantics}, the input-output model $m_{inter}$ is the output of executing a given layer that is passed onto the next layer as input. Note that an empty input-output model ($\epsilon$) is passed as the second argument to the $layerstep$ relation in \cref{def:transformation_step_semantics}. This is because in \cref{def:layer_step_semantics} of layer step semantics, the second argument of the relation is used as an accumulator for the model fragments that are added to be added to the output of the previous layer once all the rules in the current layer have been executed. The transformation rules in a layer are expanded before execution ($l^{\star}$) such that polymorphism in the match elements can be handled (see \cref{def:transformation_rule_expansion_appendix}). 

\begin{definition} {Model Transformation Execution}
\label{def:modeltransformation_appendix} 

\CatchFileBetweenTags{\modeltransformation}{text/definitions}{modeltransformationappendix}{\modeltransformation}
\end{definition}

\CatchFileBetweenTags{\modeltransformationtextappendix}{text/definitions}{modeltransformationtextappendix}{\modeltransformationtextappendix}

% A transformation execution is formed from taking the input model, and executing
% the transformation on it to produce the output model. During this
% transformation, traceability links will be placed between match elements and the
% apply elements they create. Definition~\ref{def:transf_ex} expresses this
% formally. Note that we assume that transformation executions are built following
% the algorithm described in~\cite{DBLP:conf/sle/BarrocaLAFS10}.

% \begin{definition}{Transformation Execution}
% \label{def:transf_ex}
% 
% Let $t$ be a DSLTrans transformation having source metamodel $s$ and target
% metamodel $t$. A Transformation Execution is a 6-tuple $\langle V,E\cup
% Tl,\tau,Match,Apply,Tl\rangle$, where $\langle V,E,\tau,Match,Apply\rangle \in
% MAP^{s}_{t}$ is a match-apply pattern. $Match=\langle V',E', \tau',s\rangle$,
% $Apply=\langle V'',E'', \tau'',t\rangle$ and the edges $Tl\subseteq V'\times
% V''$ are called \emph{traceability links}. The set of all transformation
% executions having source metamodel $s$ and target metamodel $t$ is called
% $Exec^{s}_{t}$.
% \end{definition}

We now prove two important properties about executions of transformations expressed in the subset of DSLTrans presented in this paper: \emph{confluence} and \emph{termination}. The proofs are provided at a high level, given the fact that DSLTrans essentially enforces both these properties by construction of the semantics of DSLTrans.

\begin{proposition}{Confluence}

Every model transformation execution is confluent up to typed graph isomorphism.
\end{proposition}
\begin{pf}
We want to prove that for every model transformation execution of a transformation $tr\in \textsc{Transf}^{sr}_{tg}$ having as input a model $input \in MODEL^{s}$, its output is always the same up to typed graph isomorphism.\\
If we assume an execution of the transformation is not confluent then this should happen because of non-determinism when the execution of a transformation is being built. Non-determinism happens during the construction of a transformation execution at two points: 
\begin{enumerate}
\item in definition~\ref{def:apply_function}, $a_{\Delta}$ is non-deterministic up to typed graph isomorphism, which does not contradict the proposition we are trying to prove.
\item in definition~\ref{def:layer_step_semantics} transformation rule $rl$ is chosen non-deterministically from layer $l$. Thus, the order in which the transformation rules are treated is non-deterministic. However, by \cref{def:layer_step_semantics} rules in a layer execute independently, which means no-side effects from rule ordering influence the execution of rules in the same layer. Also, the increments to the transformation by each rule of a layer in \cref{def:layer_step_semantics} are united using $\sqcup$ (see \cref{def:typed_graph_union_appendix}), which is an operation that is commutative by construction and thus renders the transformation result from each layer deterministic. 
\end{enumerate}
Given there are no other sources of non-determinism when building the execution of a transformation, every model transformation execution is confluent up to typed graph isomorphism.
\end{pf}

\begin{proposition}{Termination}

Every model transformation execution terminates.
\end{proposition}
\begin{pf}
Let us assume that there is a transformation execution which does not terminate. In order for this to happen there must exist a part in the construction of the execution of a transformation which induces an algorithm with an infinite amount of steps. We identify three moments when this can happen:
\begin{enumerate}
\item if definition~\ref{def:layer_step_semantics} of execution of a layer induces an infinite amount of steps. The only possibility for this to happen is if a layer has an infinite amount of transformation rules, which is a contradiction with definition~\ref{def:layer_transformation_appendix}.
\item if definition~\ref{def:transformation_step_semantics} of execution of a transformation induces an infinite amount of steps. Given layers are executed sequentially and no looping is allowed, the only possibility for this to happen is if the transformation has an infinite amount of layers, which contradicts definition~\ref{def:layer_transformation_appendix}.
\item if the result of the $match_{rl}(m_{in})$ function in definition~\ref{def:match_function} is an infinite set of match-apply graphs. The input-output model $m_{in}$ is by definition finite and the matching of each rule is independent from the execution of other rules in the same layer. As such, the number of subgraphs of $m_{in}$ isomorphic to $rl$'s matcher found during the execution of $rl$ is finite.
\end{enumerate}
Given there are no other constructs in the semantics of a transformation that can induce an infinite amount of steps, every model transformation execution terminates.
\end{pf}

