\section{Formal Background}
\label{sec:appendix_formal_background}

\counterwithin{definition}{section}
\setcounter{definition}{0}
\setcounter{subsection}{0}
\onehalfspacing 

\CatchFileBetweenTags{\typedgraphtext}{text/definitions}{typedgraphtext}{\typedgraphtext}

\begin{definition}{Typed Graph\\}
\label{def:typed_graph_appendix}
\CatchFileBetweenTags{\typedgraph}{text/definitions}{typedgraph}{\typedgraph}
\end{definition}


\CatchFileBetweenTags{\typedgraphuniontext}{text/definitions}{typedgraphuniontext}{\typedgraphuniontext}

\begin{definition}{Typed Graph Union\\}
\label{def:typed_graph_union_appendix}
\CatchFileBetweenTags{\typedgraphunion}{text/definitions}{typedgraphunion}{\typedgraphunion}
\end{definition}


\CatchFileBetweenTags{\typedgraphhomomorphismtext}{text/definitions}{typedgraphhomomorphismtext}{\typedgraphhomomorphismtext}


\begin{definition}{Typed Graph Homomorphism\\}
\label{def:typed_graph_homomorphism_appendix}
\CatchFileBetweenTags{\typedgraphhomomorphism}{text/definitions}{typedgraphhomomorphism}{\typedgraphhomomorphism}
\end{definition}

Note that, trivially, a typed graph homomorphism is a graph homomorphism.

\CatchFileBetweenTags{\typedsubgraphtext}{text/definitions}{typedsubgraphtext}{\typedsubgraphtext}

\begin{definition}{Typed Subgraph\\}
\label{def:typedsubgraph_appendix}
\CatchFileBetweenTags{\typedsubgraph}{text/definitions}{typedsubgraph}{\typedsubgraph}
\end{definition}

\CatchFileBetweenTags{\typedgraphisomorphismtext}{text/definitions}{typedgraphisomorphismtext}{\typedgraphisomorphismtext}

\begin{definition}{Typed Graph Isomorphism\\}
\label{def:typed_graph_isomorphism_appendix}
\CatchFileBetweenTags{\typedgraphisomorphism}{text/definitions}{typedgraphisomorphism}{\typedgraphisomorphism}
\end{definition}

\paragraph{\textbf{Notation:}}
\CatchFileBetweenTags{\notationformalbackground}{text/definitions}{notationformalbackground}{\notationformalbackground}

\clearpage
