\section{Definitions and Proofs of the Denotational Semantics}
\label{sec:proofs_denotation}

\subsection{Basic Definitions}

\begin{definition}{Pattern}
\label{def:rule_pattern}
A \emph{pattern} over a given set of vertex types $\mathcal{VT} = \{ \mathcal{V}_1,\dots,\mathcal{V}_k\}$ and a given set of edge types $\mathcal{ET} = \{ \mathcal{E}_1, \dots,\mathcal{E}_l \} \subseteq \{ (\mathcal{V}_i \times \mathcal{V}_j \mid 1 \leq i,j \leq k \}$ is defined as the structure $(V_1,\dots,V_m,E_1,\dots E_n)$ with $V_i \in \mathcal{VT}$ for $i \leq i \leq m$ and $E_i \in \mathcal{ET}$ for $1 \leq i \leq n$. It represents an instantiable graph pattern by describing the vertices and edges used in that rule and the set of their possible instances.
\end{definition}

\begin{example}{Pattern}
For the Stations Rule in Figure \ref{fig:dsltransformation}, the corresponding pattern is
$(S_O,S_G,T)$ where $S_O$ represents the set of all station instances from the orgnization language, $S_G$, the corresponding set of the gender language, and $T$ the set of a trace links between a pair of instances.
\end{example}

\begin{definition}{Instance Graph}
\label{def:instance_graph}
An \emph{instance graph} $(V,E)$ with $V = \{ v_1,\dots,v_m \}$ where $v_i \in V_i$ for $1 \leq i \leq m$ and $v_i \neq v_j$ for $i \neq j$, and $E = \{ e_1,\dots, e_n \}$ where $e_i \in E_i$ for $1 \leq i \leq n$ and $e_i \neq e_j$ for $i \neq j$ of a \emph{pattern} represents a graph instantiating a pattern, respecting the type constraints and multiplicities defined by that pattern.
\end{definition}

\begin{example}{Instance Graph}
Again, using the Stations Rule in Figure \ref{fig:dsltransformation}, the set of corresponding instance graphs is $\{ ( \{ s_o, s_t \} , \{  \} ) \mid \}$
$(S_O,S_G,T)$ where $S_O$ represents the set of all station instances from the orgnization language, $S_G$, the corresponding set of the gender language, and $T$ the set of a trace links between a pair of instances.
\end{example}

\subsection{Semantics of DSLTrans}

\begin{definition}{Set of Rule Executions.}
\label{def:rule_exec_set}
Let $rl\in \textsc{Rule}^{sr}_{tg}$ be a DSLTrans rule. The set of all rule
executions is a set of graph relations, built inductively as follows:
\begin{align*}
\llbracket rl\rrbracket =\big\{&(i',o')\in \textsc{Gr}\;\lvert\;(i',o') =
(rl_{\emptyset},rl_{\emptyset})\;\lor \\&(i',o') = (o,o\sqcup rl_{pat})\land
(i,o)\in \llbracket rl\rrbracket \big\}\\\\
&\text{where } rl_{pat} \text{ is a pattern over } rl
\end{align*} 
\end{definition}

A set of rule executions is the set of all input/output pairs resulting from
executing one rule over any input model. It is built inductively by starting
from $rl_{\emptyset}$ (any graph where the rule does not match) and adding to it
any patterns of $rl$, meaning $rl$ has executed any number of times.

\levi{the definition of sets of rule executions is based on the notion of
patterns over rules from Bernhard's text. I think I got the intent of
\ref{def:rule_pattern}, but more discussion is needed such that I'm sure of my
definition.}
