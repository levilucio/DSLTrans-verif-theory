% !TEX root = ../main.tex

\section{Definitions and Proofs of the Denotational Semantics}
\label{sec:proofs_denotation}

\subsection{Basic Definitions}

\begin{definition}{Pattern}
\label{def:rule_pattern}
A \emph{pattern} over a given set of vertex types $\mathcal{VT} = \{ \mathcal{V}_1,\dots,\mathcal{V}_k\}$ and a given set of edge types $\mathcal{ET} = \{ \mathcal{E}_1, \dots,\mathcal{E}_l \} \subseteq \{ (\mathcal{V}_i \times \mathcal{V}_j \mid 1 \leq i,j \leq k \}$ is defined as the structure $(V_1,\dots,V_m,E_1,\dots E_n)$ with $V_i \in \mathcal{VT}$ for $i \leq i \leq m$ and $E_i \in \mathcal{ET}$ for $1 \leq i \leq n$. It represents an instantiable graph pattern by describing the vertices and edges used in that rule and the set of their possible instances.
\end{definition}

\begin{example}
For the Stations Rule in Figure \ref{fig:dsltransformation}, the corresponding pattern is
$(S_O,S_G,T)$ where $S_O = \{ \mbox{station}_{o,1}, \mbox{station}_{o,2},\dots \}$ represents the set of all station instances from the organization language, $S_G = \{ \mbox{station}_{g,1}, \mbox{station}_{g,2},\dots \}$, the corresponding set of the gender language, and $T = S_O \times S_G$ the set of a trace links between a pair of instances.
\end{example}

\begin{definition}{Instance Graph}
\label{def:instance_graph}
An \emph{instance graph} $(V,E)$ with $V = \{ v_1,\dots,v_m \}$ where $v_i \in V_i$ for $1 \leq i \leq m$ and $v_i \neq v_j$ for $i \neq j$, and $E = \{ e_1,\dots, e_n \}$ where $e_i \in E_i$ for $1 \leq i \leq n$ and $e_i \neq e_j$ for $i \neq j$ of a \emph{pattern} represents a graph instantiating a pattern, respecting the type constraints and multiplicities defined by that pattern.
\end{definition}

\begin{example}
Again, using the Stations Rule in Figure \ref{fig:dsltransformation}, the set of corresponding instance graphs is $\{ ( \{ s_o, s_g \} , \{ (s_o, s_t \} ) \mid s_o \in S_O \mbox{and} s_g \in S_G \}$. Thus, each instance graph consists of a pair of an organization station and a gender station, together with a single edge representing the trace link.
\end{example}

\begin{definition}{Transformation Rule}
A \emph{transformation rule} over a given set of vertex types $\mathcal{VT} = \mathcal{VT}^i \cup \mathcal{VT}^o$ with $ \mathcal{VT}^i \cap \mathcal{VT}^o = emptyset$ and a given set of edge types $\mathcal{ET} =  \mathcal{ET}^i \cup \mathcal{ET}^o$ with $\mathcal{ET}^i \cap \mathcal{ET}^o = \emptyset$ is a partitioned pattern, describing a transformation by distinguishing between input elements $ \mathcal{VT}^i$ and  $\mathcal{ET}^i$ and output elements $ \mathcal{VT}^o$ and  $\mathcal{ET}^o$. 
\end{definition}


\subsection{Semantics of DSLTrans}

\begin{definition}{Set of Rule Executions.}
\label{def:rule_exec_set}
Let $rl\in \textsc{Rule}^{sr}_{tg}$ be a DSLTrans rule. The set of all rule
executions is a set of graph relations, built inductively as follows:
\begin{align*}
\llbracket rl\rrbracket =\big\{&(i',o')\in \textsc{Gr}\;\lvert\;(i',o') =
(rl_{\emptyset},rl_{\emptyset})\;\lor \\&(i',o') = (o,o\sqcup rl_{pat})\land
(i,o)\in \llbracket rl\rrbracket \big\}
\end{align*} 
where  $rl_{pat}$ is a pattern over $rl$ and $\sqcup$
is the tracing graph union where at least one element of $\Vert rl_{pat}\Vert$
is disjoint with $o$.
\end{definition}

A set of rule executions is the set of all input/output pairs resulting from
executing one rule over any input model. It is built inductively by starting
from $rl_{\emptyset}$ (any input graph where the rule does not match) and adding
to it any traced patterns of $rl$, meaning $rl$ has executed any number of
times. Note that the match part of $rl_{pat}$ (denoted $\Vert rl_{pat}\Vert$)
can overlap with the graph in $o$ by at most $n-1$ elements, being that $n$ is the amount
of match elements in $rl_{pat}$. Traceability adds trace links between any
elements belonging all input elements and any output elements of $rl_{pat}$ that
is not connected to a backward link.

\levi{the definition of sets of rule executions is based on the notion of
patterns over rules from Bernhard's text. I think I got the intent of
\ref{def:rule_pattern}, but more discussion is needed such that I'm sure of my
definition. In particular the notion of overlap between graphs is
underspecified regarding types (is this solved in the basic definitions?)}

\begin{definition}{Set of Rule Executions for a Layer.}
\label{def:rule_exec_set}
Let $l\in \textsc{Layer}^{sr}_{tg}$ be a DSLTrans layer. The set of all layer
executions is a set of graph relations, built inductively as follows:
\begin{gather*}
  \llbracket l\rrbracket = 
  \begin{cases}
    \emptyset &\text{if}\; l=\emptyset\\
    \llbracket l_1\rrbracket \otimes \llbracket rl\rrbracket &\text{if}\;
    l = l_1\cup rl
  \end{cases}
\end{gather*}  

where $\textsc{Ex}_1 \otimes \textsc{Ex}_2$ is set of the unions of pairs of
graphs $(ex_1,ex_2)\in \textsc{Ex}_1 \times \textsc{Ex}_2$ ($\textsc{Ex}_1$ and
$\textsc{Ex}_2$ being sets of executions) where the match elements of $ex_1$ and
the match elements of $ex_2$ may be joint.
\end{definition}

\levi{this is overapproximating the set of layer executions because of the
corner case where the match of one rule contains the match of the other rule in
the layer, in which case the smaller rule will necessarily execute where the
larger one does (but not vice-versa). I think there is an elegant way to
describe this\ldots or can we just use the overapproximation as is, given it
does contain the set of all executions of the layer\ldots}
