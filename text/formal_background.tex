\section{Formal Background}
\label{sec:formal_background}

In this section we will introduce formal structures and concepts that will be used throughout this paper. We start in Section \ref{subsec:typed_graphs} by a few (typed) graph concepts that will be used as mathematical building blocks. In particular we introduce our representation of typed graphs and useful relations between typed graphs based on homomorphisms. Notice that these concepts are well known from graph theory and we attempt to make minimal customizations for our purposes.

This section is based on that of \cite{Barroca2013},

Armed with these fundamental definitions, we can then introduce the formal concepts of the DSLTrans transformation language in Section~\ref{subsec:dsltrans_formal_structures}, and the symbolic representation of transformation executions in Section~\ref{subsubsec:path_condition_creation}.


\newcommand{\defineggprime}{Let $\langle V,E,(s,t),\tau,VT,ET\rangle = g$, and\\ $\langle V',E',(s',t'),\tau',VT',ET'\rangle = g'$, where $g, g' \in \textsc{TG}$.}

\newcommand{\ET}{\mathit{ET}}
\newcommand{\VT}{\mathit{VT}}


\bentley{Make sure that terminology is consistent between elements and vertices}

\subsection{Typed Graphs}
\label{subsec:typed_graphs}


\subsubsection*{Typed Graph}
We will start by introducing the notion of typed graph. A typed graph is the essential object we will use throughout our mathematical development. Typed graphs will be used to formalise all the important graph-like structures we will present in this paper. A typed graph is a directed multigraph (a graph allowing multiple edges between two vertices) where vertices and edges are typed.


\begin{definition}{Typed Graph\\}
\label{def:typed_graph}
A typed graph is a 6-tuple $\langle V,E,(s,t),\tau, \VT, \ET\rangle$ where:
\begin{itemize}
\item $V$ is a finite set of vertices
\item $E$ is a finite set of directed edges connecting the vertices $V$
\item $(s,t)$ is a pair of functions $s: E\rightarrow V$ and $t: E\rightarrow V$ that respectively provide the source and target vertices for each edge in the graph
\item Function $\tau:V\cup E\rightarrow \VT \cup \ET$ is a typing function for the elements of $V$ and $E$, where $\VT$ and $\ET$ are disjoint finite sets of vertex and edge type identifiers and $\tau(v)\in \VT$ if $v\in V$ and $\tau(e)\in \ET$ if $e\in E$
\item Edges $e\in E$ are noted $v\xrightarrow{e} v'$ if $s(e)=v$ and $t(e)=v'$, or simply e if the context is unambiguous
\item The set of all typed graphs is called $\textsc{TG}$
\item We define the empty graph to be a graph with all elements to be empty functions or sets
\end{itemize}
\end{definition}


\subsubsection{Vertex and Edge Types}

Note that our verification technique is geared toward model-driven engineering. Therefore, the types of vertices and edges in our typed graphs will be drawn from a particular \textit{metamodel}. An example transformation with source and target metamodels is presented in Section~\ref{subsec:fam_to_persons_example}. Note that DSLTrans rules (to be formally defined) will consist of multiple typed graphs with potentially different metamodels.

We assume that this metamodel provides the necessary vertex and edge types, and that all duplicate types have been resolved. 
%As well, we assume that any name collisions arising from duplicate names in the source and target metamodel is resolved by appending the metamodel name to the vertex or edge type name.
As well, we assume that the metamodel provides a partial ordering $\leq$ on the vertex and edge types for subtyping information.

For convenience, we define some utility functions in our formalization to aid the treatment of typing:

\begin{itemize}
\item $\mathit{isAbstract}: \mathit{VT} \rightarrow \{\mathit{true}$, $\mathit{false}\}$
\begin{itemize}
\item Where $\mathit{isAbstract}(\VT)$ returns $\mathit{true}$ iff $\VT$ is denoted as abstract (not able to be instantiated) by the metamodel, else $\mathit{false}$
\end{itemize}

\item $\mathit{isIndirect}: \mathit{ET} \rightarrow \{\mathit{true}$, $\mathit{false}\}$
\begin{itemize}
\item Where $\mathit{isIndirect}(\ET)$ returns $\mathit{true}$ iff $\ET$ is denoted as an \textit{indirect edge}, else $\mathit{false}$. The \textit{indirect} classification allows matching over indirect paths between vertices. This will be further clarified when required for our constructs.
\end{itemize}

\item $\mathit{matchesOver}: \{\VT \cup \ET\} \times \{\VT \cup \ET\} \rightarrow \{\mathit{true}$, $\mathit{false}\}$

\begin{itemize}
\item The purpose of this function is to assist in handling polymorphism in the metamodel, and to handle matching for the special DSLTrans types
\item $\mathit{matchesOver}(T, T) \rightarrow \mathit{true}$
\item $\mathit{matchesOver}(T, T') \rightarrow \mathit{true}$ iff T' is a subclass of T in some defined partial ordering $\leq$ given by the metamodel
\item Otherwise, $\mathit{false}$
\end{itemize}
\end{itemize}

This typing information is our implementation of a typed graph conforming to a metamodel. Note that for simplification purposes, we will not represent edge cardinalities or containment relationships given by a metamodel in our notion of typed graph. In fact we require these conditions to be relaxed to perform our graph rewriting.

\subsubsection*{Typed Subgraph}
We now define the useful notion of typed subgraph. As expected, a typed subgraph is simply a restriction of a typed graph to some of its vertices and edges. This will be needed to partition DSLTrans constructs such as rules into logical components.

\begin{definition}{Typed Subgraph\\}
\label{def:typedsubgraph}
\defineggprime

$g'$ is a typed subgraph of $g$, written $g'\sqsubseteq g$, iff $V'\subseteq V$, $E'\subseteq E$ and $\tau'=\tau |_{V'\cup E'}$.

\end{definition}

%Given a typed graph $g\in \textsc{TG}$, will use the notation $Components(g)$ to describe the set of strongly connected typed graphs in $g$. Also, we will use the notation $g|_{t}$ to refer to the restriction of graph $g$ to its subgraph containing only edges of type $t$.

We also define a function getEdges: \textit{ET} $\rightarrow$ Set(Edges) to return all edges of a certain edge type.

\bentley{Where is this used?}

\subsubsection*{Typed Graph Union}
We now define how two typed graphs are united. A union of two typed graphs is trivially the set union of all the components of those two typed graphs. Note that we do not require the components of the two graphs to be disjoint, as in the following joint unions will be used to merge typed graphs.

\begin{definition}{Typed Graph Union\\}
\label{def:typed_graph_union}

The typed graph union is the function $\sqcup :\textsc{TG}\times \textsc{TG}\rightarrow \textsc{TG}$ defined as:
\begin{multline*}
\big\langle V,E,(s,t),\tau,\VT,\ET\big\rangle\;\sqcup\;\big\langle V',E',(s',t'),\tau',\VT',\ET'\big\rangle=\\
\big\langle V\cup V', E\cup E',(s\cup s', t\cup t'), \tau\cup \tau', \VT\cup \VT', \ET\cup ET'\big\rangle
\end{multline*}
\end{definition}

Note: as a reviewer helpfully pointed out, we require that $s \cup s'$, $t \cup t'$, and $\tau \cup \tau'$ coincide on common elements. However, this can be assumed w. l. o. g..


\bentley{Do we need typed graph intersection? Or to mention what it means for two graphs to be disjoint?}
\bentley{Is this used in the DPO approach?}

\subsection{Homomorphisms}
\label{subsec:homomorphisms}

For the formal development of our technique, we are interested in relations between typed graphs that preserve some graph structure and vertex/edge type, i.e. homomorphisms.

However, our technique of symbolic execution of transformation rules means that we cannot rely on a trivial homomorphism. Instead our matching must be flexible about how vertices in the pattern graph may match over multiple vertices in the target graph. The motivation for requiring this `one-to-many' matching is provided in Section~\ref{sec:abstraction_relation}.

\subsubsection*{Typed Graph Homomorphism}

The first typed graph homomorphism we define is standard in the literature, where the structure of the pattern graph is found one-to-one in the target graph. This homomorphism is presented for the reader to appreciate the complications required for the non-standard homomorphism.

\begin{definition}{Typed Graph Homomorphism\\}
\label{def:typed_graph_homomorphism}
\defineggprime

A typed graph homomorphism from $g$ onto $g'$ is a function $f: f_v \cup f_e$ such that:
\begin{itemize}
\item $f_v: V\rightarrow V'$
\item $f_e: E\rightarrow E'$
\item $\forall v_1 \xrightarrow{e} v_2\in E$:
\begin{itemize}
\item Let $ v_1' = f_v(v_1), v_2' = f_v(v_2), e' = f_e(e)$
\item $v_1', v_2' \in V', e' \in E'$
\item $s'(e') = v_1'$, $t'(e') = v_2'$
\item $\mathit{matchesOver}(\tau(v_1), \tau(v_1') \land \\\mathit{matchesOver}(\tau(v_2), \tau(v_2') \land\\ \mathit{matchesOver}(\tau(e), \tau(e')$
\end{itemize}

%, where $\tau(v_1)=\tau'(f(v_1))$, $\tau(v_2)=\tau'(f(v_2))$ and also $\tau(e)=\tau(e')$.
%\item The domain of $f$ is noted $Dom(f)$ and the co-domain of $f$ is noted $CoDom(f)$.
\end{itemize}  
\end{definition}

 When an \emph{injective}\footnote{For a terminology refresh, \textit{injective matching} means there is a one-to-one correspondence between the pattern and target vertices. \textit{Surjective matching} means that all target vertices must be matched by at least one pattern vertex. A \textit{bijection} is when the matching is both injective and surjective.} typed graph homomorphism $f$ exists from $g$ onto $g'$, we write $g \stackrel{inj}{\blacktriangleright} g'$. When a \emph{surjective} typed graph homomorphism $f$ exists from $g$ onto $g'$ we write $g \stackrel{surj}{\blacktriangleright} g'$.

\subsubsection*{Typed Graph Edge Homomorphism}

Our technique also requires a form of graph homomorphism that primarily focuses on matching the edges of the graphs. This homomorphism will be an injective match regarding the edges, but note that a particular vertex in the pattern graph may match onto multiple vertices in the target graph. Thus, this homomorphism is `one-to-many'.

\begin{definition}{Typed Graph Edge Homomorphism\\}
\label{def:typed_graph_edge_homomorphism}
\defineggprime

A typed graph edge homomorphism between $g$ and $g'$ is a function $h: h_v \cup h_e$ such that:
\begin{itemize}
\item $h_v: V'\rightarrow V$ Note that this function has unusual properties:
\begin{itemize}

\item A function from a vertex in the target graph onto a vertex in the pattern graph
\item A partial function, as not all vertices in the target graph will be reflected in the pattern graph
\item Surjective, so that all pattern vertices are matched to by target vertices
\item A minimal function, such that the minimum number of target vertices are matched to each pattern vertex \bentley{Not sure if this is needed.}
\end{itemize}
\item $h_e: E\rightarrow E'$, and is injective
\item $\forall v_1 \xrightarrow{e} v_2\in E$:
\begin{itemize}
\item Let $e' = h_e(e), v_1' = s'(e'), v_2' = t'(e')$
\item $v_1', v_2' \in V', e' \in E'$
\item $v_1 = h_v(v_1'), v_2 = h_v(v_2')$
\item $\mathit{matchesOver}(\tau(v_1), \tau(v_1') \land \\\mathit{matchesOver}(\tau(v_2), \tau(v_2') \land\\ \mathit{matchesOver}(\tau(e), \tau(e')$
\end{itemize}
 
\end{itemize}  
\end{definition}

Again, we wish to highlight the fact that this homomorphism is different than Definition~\ref{def:typed_graph_homomorphism} in the vertex matching function. A vertex in the target graph matches onto a vertex in the pattern graph, with multiple target vertices matching onto the same pattern vertex. 
\bentley{Create diagram to illustrate}

This unusual homomorphism allows our technique to `split' our pattern graph over multiple locations in the target graph. This is critical, as our target graphs are constructed in a way that allows for vertex duplication. That is, two vertices of a particular type in the target graph may only represent one underlying element, as discussed in Section~\ref{sec:abstraction_relation}. Constructing this morphism allows our prover implementation to avoid explicitly creating `disambiguated' graphs where these duplications are resolved.

 When an \emph{injective} typed graph edge homomorphism $f$ exists from $g$ onto $g'$, we write $g \stackrel{inj}{\vartriangleright} g'$. When a \emph{surjective} typed graph edge homomorphism $f$ exists from $g$ onto $g'$ we write $g \stackrel{surj}{\vartriangleright} g'$.



\subsection*{Typed Graph Isomorphism}
Two typed graphs are said to be isomorphic if they have exactly the same shape and related vertices and edges have the same type.


\begin{definition}{Typed Graph Isomorphism\\}
\label{def:typed_graph_isomorphism}
\defineggprime

We say that $g$ and $g'$ are isomorphic, written $g\cong g'$, iff there exists a bijective typed graph homomorphism $f:V\rightarrow V'$ such that $f^{-1}:V'\rightarrow V$ is a typed graph homomorphism.
\end{definition}




\subsubsection*{Transitive Closure}

DSLTrans matching constructs allow for the matching of indirect links, where as described in Section~\ref{subsec:DSLTrans_constructs}, rules may match over indirect paths between elements. This transitive closure allow for the explicit creation of all edges implied by these indirect links to aid in matching.


\begin{definition}{Transitive Closure\\}
\label{def:instance_closure}
Let $g$ be a graph $\langle V,E,(s, t),\tau\rangle \in \textsc{TG}$. Then the transitive closure $g^{*} = \langle V,E',(s', t'),\tau', \VT, \ET' \rangle \in \textsc{TG}$ is created such that:

\begin{itemize}
\item $E' = E \cup E_{tc}$ where each new edge $e_{tc} \in E_{tc}$ is created between vertices $v_1$ and $v_2$ in $V$:
\begin{itemize}
\item $v_1\xrightarrow{e_{tc}}v_2 | \exists v_1\xrightarrow{e_i}v_i, v_i\xrightarrow{e_j}v_j, \dots, v_k\xrightarrow{e_k}v_2$
\end{itemize}

\item $s'(e_{tc}) = v_1$, $t'(e_{tc}) = v_2$
\item $\tau(e_{tc}) = \mathit{indirect}$
\end{itemize}

\end{definition}


Given a graph, its transitive closure includes, besides the original graph, all the edges belonging to the transitive closure of indirect links in that graph. Note that these transitive edges are given a special type \textit{indirect}, and will be identified by the function $\mathit{isIndirect}$.

In the definitions that follow we will use the $*$ notation, as in \cref{def:instance_closure}, to denote the transitive closure of our structures.


\subsubsection{Link Homomorphism}

For convenience, we also define a function to match components in our structures which contains only links. An example would be the backward link construct for DSLTrans rules, as discussed in Section~\ref{subsec:dsltrans_formal_structures}.

\begin{definition}{Link Homomorphism\\}
\label{def:link_homomorphism}
Consider two sets of links $A = \{E_{A}, (s_{A}, t_{A})\}$, and $B = \{E_{B}, (s_{B}, t_{B})\}$. Note that these could be backward links or traceability links.

We define a (trivial) injective homomorphism function $f$ to match $A$ over $B$, such that:

\begin{itemize}
\item $\forall e_A \in E_A:$
\begin{itemize}
\item $\exists e_b \in E_{B} | f(e_A) = e_b$
\item $f(s_{A}(e_A)) = s_{B}(e_B)$
\item $f(t_{A}(e_A)) = t_{B}(e_B)$
\end{itemize}
\end{itemize} 
\end{definition}

Note that the typing of these links will be given by the component in which they are found.

\subsubsection{Homomorphisms Between Structures}
\label{subsubsec:structure_homomorphism}

In the developments that follow, we will need to find morphisms between different structures.

For example, we may need to find a homomorphism $m$ between two structures $S_1 = \langle A, B, L\rangle$ and  $S_2 = \langle A', B', L'\rangle$, where $A, A', B, B' \in TG$ and $L, L'$ are links of the form $\{E, (s, t)\}$.

This overall homomorphism be composed of the sub-homomorphisms $f(A) = A'$, $g(B) = B'$, $h(L) = L'$, where $f$ and $g$ are the homomorphisms to be found, and $h$ is the link homomorphism. Note that $f$ and $g$ may have differing properties, such as being injective or not, motivating our consideration of three sub-morphisms.

Our definitions for these structures require that the components (such as $A$, $B$, and $L$) are all disjoint. Thus we consider the homomorphism to be trivially composed: $m = f \cup g \cup h$. 

However, a reviewer pointed out that we also require that these sub-homomorphisms are consistent with each other. Namely that the link homomorphism $h$ must agree with the bindings given by $f$ and $g$. That is:

\bentley{Finish}



