\section{\uppercase{Introduction}}
\label{sec:intro}

Model transformations were described as the \emph{heart and soul} of model driven
software development (MDD) by Sendall and Kozaczynski in
2003~\cite{Sendall2003}.  Due to their practicality and appropriate level of
abstraction, model transformations are the current technique for performing
computations on models. In their well-known 2006 paper `A Taxonomy of Model
Transformations', the authors Mens, Czarnecki and Van Gorp call for the
development of verification, validation and testing techniques for model
transformations~\cite{Mens2006125}. Despite the many publications on this topic
since then, the field of analysis of model transformations seems to be still in
its (late) infancy, as evidenced by Amrani, L\'ucio \emph{et al.}~\cite{DBLP:conf/icst/AmraniLSCDVTC12}.

% Since then, many model transformation languages have emerged and are being
% used intensively not only in MDD, but also in other scopes, such as when
% software development processes benefit from formalised translators between
% languages.

In this paper, we present our work on verification of properties on model
transformations. Specifically, we discuss concrete algorithms that can prove
whether properties will hold or do not hold on all executions of a
transformation written in the DSLTrans transformation language. Properties are
proved through a process that constructs a set of path conditions, where
each path condition symbolically represents an infinite number of concrete transformation
executions through an \emph{abstraction relation}.

% The technique has been extended from its original presentation
% in~\cite{Lucio:10}, ~\cite{DBLP:conf/sle/BarrocaLAFS10}, and~\cite{LucVan:13}.

In our previous work~\cite{Lucio:10}, this property-proving algorithm was
presented as a proof-of-concept. In the present work, we significantly expand
that proof-of-concept by clarifying and offering discussions on validity and
completeness for the presented algorithms. We also provide an implementation
that we believe will scale to industrial applications, as validated by an
automotive case study~\cite{gehan:13}.
 


Our approach is feasible due to the use of the transformation language DSLTrans~\cite{DBLP:conf/sle/BarrocaLAFS10}. DSLTrans is
\emph{Turing incomplete}, as it avoids constructs which imply unbounded
recursion or non-determinism. Despite this \emph{expressiveness reduction}, we have shown via several
examples~\cite{febavava:10,dsltrans_manual,zhang:ACP_APN:11} that DSLTrans is
sufficiently expressive to tackle typical translation problems. This sacrifice
of Turing-completeness allows us to construct a\\provably-finite set of path
conditions~\cite{Lucio:10}. Our approach currently considers a core subset
of the DSLTrans language that does not include negative conditions in rules or attribute manipulation. These features of the language will be addressed in future work.

Informed by the structure of DSLTrans transformations, our approach defines an algorithm for the creation of path conditions. Each path condition that is created represents a set of concrete
transformation executions through an \emph{abstraction relation} that we formally define. Once the set of all path conditions has been created for the transformation, we can then prove structural \emph{model syntax
relations}~\cite{DBLP:conf/icst/AmraniLSCDVTC12} using this relation. Such properties are essentially pre-condition/post-condition axioms involving statements about whether certain elements of the input model have been correctly transformed into elements of the output model, and have been explored by several authors~\cite{Akehurst:02,Narayanan:07,DBLP:journals/eceasst/CariouBBD09,DBLP:journals/ase/GuerraLWKKRSS13}.
In our proof technique, the properties examined can be proven to hold for all executions of a given
model transformation, no matter the input model.
Therefore, our technique is \emph{transformation dependent} and \emph{input
independent}~\cite{DBLP:conf/icst/AmraniLSCDVTC12}.

Our methods differ from previous work in the transformation verification field
in that we require no intermediate representation for a specific proving
framework (as in~\cite{ButtnerEC12,ButtnerECG12,DBLP:conf/icst/AsztalosLL10}) but instead work on DSLTrans
transformations themselves. Along with DSLTrans rules, all of the constructs
involved in our algorithms are typed graphs. This intuitive representation allows
our property proving technique to be composed of relatively simple steps, as the
metamodels, models, and properties involved are all constructed using a similar
graphical representation.

\reviewer{On Page 2, left, mid-page you state that your methods “differ from previous
work”in that you require no intermediate representation. It is certainly not preor-
dained that this difference is an advantage: such an intermediate representation
typically gives access to general-purpose tooling, e.g., for model checking or
theorem proving. You would have to argue that you are better off reimplement-
ing that functionality in your own setting. Is there any evidence for that? An
example of such evidence (in a somewhat different context) is for instance pro-
vided by Pennemann in his PhD thesis, where by experiments he shows that his
dedicated, graph-based resolution outperforms that of a general-purpose theorem
prover.}
% This also provided us with performance gains over our
% previous implementation, due to the use of efficient graph-matching libraries to manipulate the graph structures required.

A large difficulty in any exhaustive proof technique is the tendency for the
state space to explode, even when abstractions are performed to render the
search space finite.  A later section of this work will discuss optimisation
opportunities and performance results obtained from our implementation. The
scalability of our approach will also be analysed in order to infer the
algorithm's potential applicability to real-world problems. A real-world industrial case study will also be briefly presented.

Our specific contributions include:

\begin{itemize}
\item An algorithm for constructing all path conditions for a given DSLTrans transformation;
\item An algorithm that proves transformation properties over these path conditions;
\item Validity and completeness proofs of the path condition construction and property proving algorithms;
\item A discussion of performance and scalability results for our implementation.
\end{itemize}

This paper is organised as follows: \cref{sec:dsltrans} briefly introduces
the DSLTrans model transformation language and its formal semantics, while the formal background for this work is presented in \cref{sec:formal_background}. The algorithms to build the complete set of path
conditions for the transformation will be discussed in
\cref{sec:building_pcs}. \cref{sec:abstraction_relation} will present the abstraction relation found in our technique, along with examples, while \cref{sec:verif_dsltrans_props} will examine how this abstraction relation is used in our process for proving properties. In \cref{sec:implementation} and \cref{sec:experiments}, we introduce our implementation with sample scalability results; \cref{sec:related_work} presents the related work;
and finally in \cref{sec:conclusion} we conclude with remarks and future
work.

\reviewer{The paper is inconsistent (maybe due to the iteration of submitted versions) in
what it says about its own structure. Page 2, right, 1st paragraph states that the
language and its samentcs are presented in Section 2; Section 2 (1st paragraph)
says that the semantics is presented in Appendix B; most of it is actually in
Section 3. Appendix B indeed repeats essentially all of Section 3, a duplication
which certainly serves no purpose and should be remedied. In fact, I would much
prefer all of Appendix B to be moved to the main text, as the only part that is
now only in the appendix, the definition of a transformation, is absolutely central
to the paper.
As it currently stands, Section 3.4 has a similar problem: first it states that the
semantics is not in the main text, then it refers to Section 3.2 for the semantics
(which the reader has just gone through), then in Def. 17 jumps to a notation that
is only introduced in Appendix B.}
